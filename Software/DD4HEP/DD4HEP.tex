\section{DD4HEP}

\subsection{Introduction}
DD4hep\cite{dd4hepWebsite,1742-6596-513-2-022010} provides a generic, consistent and complete detector description, including geometry, materials, visualization, readout, alignment and calibration. It supports the full experiment life cycle: from detector concept development over detector optimization and construction to the operation phase. A single source of information is used for simulation, reconstruction and analysis, where different interfaces and formats are provided as needed. DD4hep is implemented using the ROOT geometry package TGeom.

\subsection{Recent Milestones}
The core of DD4hep provides the necessary code and tools for a complete and flexible detector description, based on C++ classes per sub detector and corresponding XML files holding parameters. It provides a palette of simple and generic sub detector geometry classes, which allows new users to get started very quickly with using DD4hep by simply adapting the XML parameters files to define a new particle physics detector. Advanced users can write their own detector descriptors to incorporate any level of detail that is needed. A complete toolkit (DDG4)\cite{DDG4:CHEP2015} for running a Geant4 based detector simulation based on a DD4hep detector model has been developed. It provides software modules for fully configuring and running a simulation application, including reading of generator files in various formats, overlaying several events, linking Monte Carlo truth information to hits and creation of the final output files in the LCIO file format. The programs can either be run as a python application or a C++ application with XML configuration files. The current Mokka simulation models for ILD have been fully ported to DD4hep (in the lcgeo package) and the CLICdp group describes their new detector model exclusively in DD4hep/lcgeo.

\subsection{Engineering Challenges}
One of the most challenging aspects in the implementation of DD4hep lay in hiding some of the complexity and technicalities involved in detailed geometry models from the user in order to facilitate the development of maintainable experiment detector description code. A considerable fraction of the complexity is created by the fact that ROOT and Geant4 have independent implementations of the geometry classes, which partly differ in constructor arguments (meaning and order) as well as in a different set of units used in the two systems. Another challenge will be to make DD4hep compatible with multi-threading applications for simulation and reconstruction/analysis.

\subsection{Future Plans}
The improvement of the core functionality as well the development of new features in DD4hep will continue over the next years. Besides addressing the multi-threading needs of the community, an interface to conditions and alignment data is on the list of extension projects already identified. for DD4hep. Additional requirements and requests brought forward by the user community will have to be addressed.

\subsection{Applications Outside of Linear Colliders}
DD4hep has been designed from the start as a generic tool that can be applied to any particle physics experiment. It is currently used by the ILD and CLICdp detector concepts as the main source of detector geometry information and simulation application (based on DDG4). The three FCC studies (FCC-ee, FCC-hh, FCC-eh) have recently also decided to base their software chain on DD4hep and will use it for the conceptual design reports in the next years.
