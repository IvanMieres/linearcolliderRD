\thispagestyle{empty}
\newgeometry{margin=1.5cm} % modify this if you need even more space
\begin{landscape}
    \centering
    \begin{adjustbox}{max width=1.1\textwidth,totalheight=1\textheight}
\begin{tabularx}{2\textheight}{lXXXX}
    \toprule
    R\&D Technology & Participating Institutes & Description / Concept & Milestones & Future Activities \\
    \midrule
        ChronoPix &
        University of Oregon\newline Yale University\newline Sarnoff Corporation &
        ChronoPix is a monolithic CMOS pixelated sensor with the ability to record up to two time stamps per pixel during the bunch train. Hits are read out in the time between bunches. &
        April 2014: Device tests of prototype 2 inform the design of prototype 3 to be submitted to foundry &
        Prototype 3 was manufactured in September 2014. Tests have shown that problems revealed in prototype 2 were solved. \\
    \midrule
        CMOS MAPS &
        IPHC Strasbourg \newline DESY, Hamburg \newline University of Bristol \newline University of Frankfurt &
        The CMOS pixel sensor uses as a sensitive volume the \SIrange{10}{20}{\micro\meter} thin high-resistivity epitaxial Si-layer deposited on low resistivity substrate of commercial CMOS processed chips. The generated charge is kept in a thin epi-layer atop the low resistivity silicon bulk by potential wells that develop at the boundary and reaches an n-well collection diode by thermal diffusion. &
        2016 : production of CPS for the ALICE-ITS upgrade\newline
        2018/19 : production of CPS for the micro-vertex detector of the CBM experiment at FAIR/GSI\newline
        2018/19 : validation of light double-sided ladder concept combining highly granular sensors on one side with timestamping sensors on the other side\newline
        $< 2020$ : validation of power pulsing of double-sdied ladders inside a high magnetic field\newline
        2022/23 : finalisation of the R\&D on various CPS adapted to the different layers of a very high performance vertex detector at the ILC &
        Until 2018-2019: Development and production of CPS for the ALICE-ITS and CBM-MVD \newline Development of various CPS optimised for the different layers of a vertex detector at the ILC, with emphasis on bunch tagging \newline Development of low material double-sided ladders \\
    \midrule
        DEPFET &
        University of Barcelona, Spain \newline
        University of Bonn, Germany \newline
        Heidelberg University, Germany \newline
        Giessen University, Germany \newline
        University of G{\"o}ttingen \newline
        KIT Karlsruhe, Germany\newline
        IFJ PAN, Krakow, Poland \newline
        MPI Munich \newline
        MPG HLL Munich, Germany \newline
        Charles University, Prague, Czech Republic \newline
        IFIC, CSIC-UVEG, Valencia, Spain \newline
        DESY, Hamburg, Germany \newline
        IFCA, CSIC-UC, Santander, Spain &
        The DEPFET technology implements a single active element within the active pixel by integrating a p-MOS transistor in each pixel on the fully depleted, detector-grade bulk silicon. Additional n-implants near the transistor act as a trap for charge carriers created in the substrate (internal gate), so that they are collected beneath the transistor gate. &
        2014: Full-scale \SI{75}{\micro\meter} thin Belle II ladder in beam test at DESY &
        Development of die-attach technology \newline
        Full-scale test of all ASICs on ladder \newline
        Integration of read-out and steering ASICs on pixel sensor using flip-chip technique and microscopic solder ball bump-bonding \newline
        Production of Belle II vertex detector modules\newline
        Tests of the last version of the DHP chips \newline
        Engineering design for all-silicon module with petal geometry required for ILC\newline
        Detailed characterization of device response \newline
        Design of ancillary ASICs, taking full responsibility for future design cycles of the FE read-out chip, called Drain Current Digitizer \\
    \midrule
        FPCCD &
        KEK \newline
        Shinshu University \newline
        Tohoku University \newline
        JAXA, Japan Aerospace Exploration Agency &
        Fine Pixel CCD sensors have pixel sizes of \SI{5}{\micro\meter} and a fully depleted epitaxial layer with a thickness of \SI{15}{\micro\meter} &
        Fabrication of real size ($\SI{12.3}{mm} \times \SI{62.4}{mm}$) sensors with \SI{50}{\micro\meter} total thickness \newline
        Neutron irradiation of a small ($\SI{6}{mm} \times \SI{6}{mm}$) FPCCD sensor \newline
        Construction of a prototype cooling system and demonstration of cooling between \SI{-40}{\degreeCelsius} and \SI{+15}{\degreeCelsius} &
        Characterization of FPCCD sensors including beam tests and radiation damage studies \newline
        Development of FPCCD sensors with a pixel size of \SI{5}{\micro\meter} \newline
        Construction of prototype ladders for the inner layers of a vertex detector \newline
        Development of readout electronics downstream of ASICs \newline
        Development of larger FPCCD sensors and prototype ladders for outer layers \newline
        Development of readout electronics with a small footprint \newline
        Construction of a real size engineering prototype and cooling test \\
    \midrule
        3D Pixels &
        Brown University \newline
        Cornell University \newline
        Fermilab \newline
        Northern Illinois University \newline
        SLAC \newline
        University of Illinois Chicago &
        3D technology allows very fine pitch (\SI{4}{\micro\meter}) integration of sensors with multiple layers of electronics, allows interconnection oto both the top and bottom of devices, and provides techniques for low mass, thinned devices. &
        Completed multi-year effort to demonstrate commercial 3D technology, consisting of \SI{0.13}{\milli\meter} CMOS interconnected with Direct Oxide bonding technology and access using TSV. \newline
        Received readout wafers with thickness of \SI{25}{\micro\meter}, processed with TSV and DBI to connect to 3D electronics \newline
        Currently working on active edge demonstrator devices &
        Complete the 3D active edge project \newline
        Apply concepts to x-ray imaging devices \newline
        Re-start ILC developments pending renewed funding \\
    \midrule
        SOI &
        KEK\newline
        University of Tsukuba \newline
        Tohoku University \newline
        Osaka University &
        In the Silicon-On-Insulator (SOI) technology the sensing and processing functionalities are separated in different layers; the sensing is provided by a high-resistive substrate connected through an insulating layer with the processing layer. &
        &
        Sep 2014: Complete architecture study for the ILC pixel detector \newline
        Mar 2015: Design and fabrication of first test chip for the ILC \newline
        Dec 2015: Beam test of the chip \\
    \midrule
        CLICPix
        &
        Cambridge University\newline
        CERN\newline
        University of Geneva\newline
        Karlsruhe Institute of Technology (KIT)\newline
        University of Liverpool\newline
        SLAC\newline
        Institute of Space Science Bucharest\newline
        Spanish Network for Linear Colliders
        &
        Hybrid pixel-detector technology comprising fast, low-power and small-pitch readout.
        ASICs implemented in \SI{65}{nm} CMOS technology (CLICpix) coupled to ultra-thin planar
        or active HV-CMOS sensors via low-mass interconnects.
        &
        Beam tests of prototype assemblies with ultra-thin sensors (\SIrange{50}{300}{\micro\meter})\newline
        CLICpix demonstrator ASIC in \SI{65}{nm} technology\newline
        Beam tests of assemblies with capacitive coupling between CCPDv3 HV-CMOS active sensors
        and CLICpix ASICs\newline
        Power-pulsing demonstrator with dummy loads\newline
        Prototypes of carbon-fibre ladder supports\newline
        Full-scale thermal mockup of the CLIC vertex-detector region
        &
        Demonstration modules for all major components
        in time for the next update of the European Strategy
        for Particle Physics in 2018/19
        \\
    \bottomrule
\end{tabularx}
\end{adjustbox}
\end{landscape}
\restoregeometry
