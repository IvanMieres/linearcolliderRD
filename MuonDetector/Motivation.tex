\section{Motivation and Constraints for Muon Detectors at Linear Colliders}

The goal of the muon detectors for the Linear Collider Detector is the identification of minimally ionizing particles. Muons do not get stopped in the calorimeters and traverse the solenoid. To measure such charged particles the solenoid field return yoke is instrumented with detectors that measure ionizing particles. The excellent track momentum resolution of Linear Collider Detector trackers means that the main requirement of the muon systems is to be able to match muon track segments with those of the inner detectors to be able to tag a track as a muon. In terms of design, this requirement translates to high efficiency and moderate segmentation.
